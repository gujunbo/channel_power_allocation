% ettdoc.tex V2.10, 19 June 2012

\documentclass[times]{ettauth}
\usepackage[toc,page]{appendix}
\usepackage{moreverb}

\usepackage[colorlinks,bookmarksopen,bookmarksnumbered,citecolor=red,urlcolor=red]{hyperref}

\usepackage{cite}
\newcommand\BibTeX{{\rmfamily B\kern-.05em \textsc{i\kern-.025em b}\kern-.08em
T\kern-.1667em\lower.7ex\hbox{E}\kern-.125emX}}

\def\volumeyear{2017}

\newcommand{\eg}{e.g., }
\newcommand{\ie}{i.e., }
\usepackage[linesnumbered,ruled,vlined]{algorithm2e}
\DeclareMathOperator*{\argmin}{\textbf{\upshape arg\,min}}
\DeclareMathOperator*{\argmax}{\textbf{\upshape arg\,max}}

\usepackage{times}
%\usepackage{mathptmx}
\usepackage{mathtools, cuted}
\usepackage[draft,nomargin,marginclue,footnote,silent]{fixme}
\setcounter{tocdepth}{2} %table of contents


\theoremstyle{mytheoremstyle}
\newtheorem{theorem}{Theorem}[section]

\theoremstyle{mytheoremstyle}
\newtheorem{corollary}{Corollary}[section]

\theoremstyle{mytheoremstyle}
\newtheorem{lemma}{Lemma}[section]

\newtheorem{mydef}{Definition}

\DeclareMathOperator*{\Max}{Max}
\DeclareMathOperator*{\Min}{Min}

\newcommand{\bigO}{\ensuremath{\mathcal{O}}}% big-O notation/symbol
\usepackage{subfigure}
\usepackage{adjustbox}
\usepackage{multirow}
\usepackage{enumitem}
\setlist[itemize]{leftmargin=*}
\setlist[enumerate]{wide=\parindent}
\usepackage[referable]{threeparttablex}
\renewlist{tablenotes}{enumerate}{1}
\makeatletter
\setlist[tablenotes]{label=\tnote{\alph*},ref=\alph*,itemsep=\z@,topsep=\z@skip,partopsep=\z@skip,parsep=\z@,itemindent=\z@,labelindent=\tabcolsep,labelsep=.2em,leftmargin=*,align=left,before={\footnotesize}}
\makeatother

\usepackage{graphicx}
\usepackage{sidecap}
\usepackage{kantlipsum} %<- For dummy text
\usepackage{mwe} %<- For dummy images
\usepackage{multirow}
\usepackage{multicol}
\usepackage{cleveref}

%% Andere Packages %%%%%%%%%%%%%%%%%%%%%%%%%%%%%%%%%%%%%%%%%%
%\usepackage{a4wide} %%Kleinere Seitenränder = mehr Text pro Zeile.
\usepackage{fancyhdr} %%Fancy Kopf- und Fußzeilen
%\usepackage{longtable} %%Für Tabellen, die eine Seite überschreiten
\usepackage{lscape}
\usepackage{rotating}
%\usepackage[htt]{hyphenat} %Trennung von Typewriter-Schriften
%\usepackage{listings}
%\usepackage{pstricks-add} --> This package generates problems with booktabs (toprule, etc.)
\usepackage[autostyle]{csquotes}
\usepackage{amsmath,bm}
\usepackage{array}
% Tabellen mit Center und left
\usepackage{tabularx,colortbl} % colored table background
%\usepackage{tablefootnote}
\newcolumntype{C}[1]{>{\centering\arraybackslash}m{#1}}
\newcolumntype{R}[1]{>{\raggedleft\arraybackslash}m{#1}}
\newcolumntype{L}[1]{>{\raggedright\arraybackslash}m{#1}}
% Table spacings
\newcommand\T{\rule{0pt}{2.5ex}\rule[-1.0ex]{0pt}{0pt}}
\newcommand\B{\rule[-1.0ex]{0pt}{0pt}}


\definecolor{slightgray}{gray}{.90}
\usepackage{rotating}
\usepackage{hhline}
\usepackage{float}
\usepackage{caption}% http://ctan.org/pkg/caption
\captionsetup[table]{format=plain,labelformat=simple,labelsep=period}
%\usepackage{authblk}

\usepackage{color}
\renewcommand{\vec}[1]{\mathbf{#1}}

\usepackage{algpseudocode}
\algnewcommand{\algorithmicand}{\textbf{ and }}
\algnewcommand{\algorithmicor}{\textbf{ or }}
\algnewcommand{\OR}{\algorithmicor}
\algnewcommand{\AND}{\algorithmicand}
\algnewcommand{\var}{\texttt}

\begin{document}

%\runningheads{A.~N.~Other}{A demonstration of the \journalabb\
%class file}

\articletype{RESEARCH ARTICLE}

\title{Distributed Schemes under Congestion Game framework and Optimization for Spectrum and Power Allocation in TVWS}
%\author[1]{Di Li\thanks{li@umic.rwth-aachen.de}}
%\author[2]{James Gross}
\author{Di Li\textsuperscript{1} James Gross\textsuperscript{2}}
%\affil[1]{RWTH Aachen University}
%\affil[3]{KTH Royal Institute of Technology}
\address{RWTH Aachen University\textsuperscript{1}, KTH Royal Institute of Technology\textsuperscript{2} }
%\renewcommand\Authands{ and }
%\corraddr{Communication Theory Lab
%School of Electrical Engineering, 
%KTH Royal Institute of Technology
%SE - 100 44 Stockholm\\Email: james.gross@ee.kth.se}
\corraddr{Chair of Communication and Distributed Systems
Ahornstrasse 55 - building E3
52074 Aachen
Germany
\\Email: li@umic.rwth-aachen.de}




\begin{abstract}
\small In this chapter, we will see the application of congestion game in solving the channel allocation problem in the context of TV white space.
The channel allocation problem we will address is a general problem, as the transmission power is not identical for every transmitter and on each channel, actually, the transmission power could be unique for each transmitter-channel combination.
With the suitable utility function designed for transmitters, the behaviours of the transmitters can be described by a congestion game.
The algorithm of channel allocation is derived from the dynamics of the transmitter in the game, which reaches Nash equilibrium quickly.

Furthermore, we provide a complete solution to fully exploit TV white space complying with IEEE 802.22 standard.
We propose a centralized methods to regulate the upper bound of transmission power, so that to strictly protect the primary users.
%Proposed scheme also considers the necessity of distributable execution which decides the working channel and transmission power.
The the distributed channel allocation and power control are conducted sequentially.
%As to the channel allocation problem, we innovatively formulate this problem in to a canonical congestion game, and design efficient distributed channel selection strategy with the assistance of the centralized database.
%The successful practice of congestion game in this problem is enlightening for the application of congestion game in other problems where asymmetric interaction exists.
\end{abstract}

%\keywords{class file; \LaTeXe; \emph{\journalabb}}




% Distributed Channel and Power Allocation in IEEE 802.22 Networks
\begin{abstract}

In this paper, we look into the problem of how to exploit the TV white space in an IEEE 802.22 like cellular network, meanwhile complying with the dominant Electronic Communications Committee (ECC) regulations which imposes additional restrictions on the usage of TV white space.
Multiple TV channels are used by the secondary users and we improve the Shanonn capacity on the end terminals.
An optimization problem is proposed and solved in centralized manner, in addition, distributed scheme is designed on the basis of congestion game.
The game theory based distributed schemes achieve comparable performance in terms of capacity on end users with more power consumption.
\end{abstract}

\maketitle
\graphicspath{
{../figures/03_distributedChannelAllocation/}{../figure/03_distributedChannelAllocation/1mar2016/}{../figure/03_distributedChannelAllocation/1mar2016/radius_6000_runtimes_50/}{../../channel-power-allocation-802.22/}{../../channel-power-allocation-802.22/plots/}
}

\section{Introduction}

%%%%%%%%%%%%%%%%%%%%%%%%%%%%%%%%%%%%%%%%%%%%%%%%%%%%%


With the transition from analog TV to digital terrestrial TV, a considerable amount of frequency bands, as shown in Figure~\ref{variability_TVWS}, become vacant.
The spectrum that is left over by digital TV and other incumbent users is referred to as TV white spaces (TVWSs).
TVWS can be used for telemedicine, precision agriculture, smart energy and so on by new devices as long as digital terrestrial TV reception is not interfered~\cite{FCC_2010_sedond_memorandumm}. 
These unlicensed devices are called white space devices and their operation should be restricted in terms of location, channel and transmission power, so that no harmful interference is disturbing the incumbents.
The utilization of TVWS by the white space devices are regulated by national regulatory authorities and relevant committees, \ie in the US, the Federal Communications Commission (FCC) has regulated the utilization of TVWS since 2010~\cite{FCC_2010_sedond_memorandumm}. 
The Electronic Communications Committee (ECC) and Ofcom have released guidelines in EU and UK respectively in terms of the operation of white space devices~\cite{ECC236, ofcom15}.
Although many regulation details from the three institutes are different, white space data base (WSDB) is adopted by all the three institutes.

TV White Space database was first introduced as a way to overcome the technical hurdles which prevent the spectrum sensing techniques from precisely detecting very weak primary signals~\cite{Mwangoka2011DySPAN}.
White space device should contact a white space data base and provide its location and technical characteristics.
The WSDB translates the information on incumbent services and the technical characteristics and location of the white space devices into a list of allowed frequencies and associated transmit powers for devices.~\cite{ECC236}.
White space device needs to access the WSDB to get the available channels and powers before starting transmission.
%WSDB is proposed mainly due to the difficulty to implement spectrum sensing, which is suggested by the regulators.%DEBUG

White space data base plays a key role in the utilization of TVWS.
It has the global view of the white space devices in the network and protects the digital terrestrial TV receivers.
The resource which is available for the white space devices is calculated by the strategy of WSDB, which is as per the regulators.
Ofcom and ECC impose restrictions on the aggregated interference caused by white space devices on the digital terrestrial TV receivers, and implicitly allow white space devices to have different transmission powers based on their locations.
FCC employs stringent restriction on the identical transmission power for white space devices, but the digital terrestrial TV receivers are not protected from the aggregated interference caused by the white space devices which work on the same channel. 
FCC's stringent regulation on transmission power results in decreased TVWS as suggested in \cite{Harrison2012Dyspan}.
%
%Nevertheless, as to the issue of spectrum sharing among the white space devices, the regulator institutes only provide suggestion instead of regulations.
In this paper, we will follow regulations issued by ECC on the usage of TVWS and investigate the spectrum sharing issue among the white space devices.




% with so called interference margin~\footnote{interference margin is the maximal interference caused by secondary users, which doesn't violate TV service.}~\cite{multipleIntf_pimrc11} which should not be exceeded by the accumulated interference caused by all secondary users working on the the channel.

%FCC and ECC have announced rules on the transmission power of secondary users working in TV white space in US and Europe respectively~\cite{FCC_2010_sedond_memorandumm, ecc159}. 
%FCC requires a minimum distance between secondary user and TV service area, besides, the transmission power for fixed secondary users is set as $4$ \textup{W}, which is a conservative setting. 
%FCC believes with these prudent measures, the interference margin can not be exceeded by interference from secondary users.
%But it may not be the case when there are multiple secondary equipments transmitting at the same time, which is pointed in~\cite{Jaentti11}.
%ECC requires the secondary users to adapt their maximum transmission power according to the distance away from the TV receivers.


\begin{figure}[h!]
  \centering
  \includegraphics[width=0.9\linewidth]{TWWS_availability_east.pdf}
  \caption{Variability of available channels in a densely populated area. This figure is obtained from \cite{googleDatabase}}
\label{variability_TVWS}
\end{figure}

The white space devices are classified into two functional categories: 1) High power, stationary stations such a base stations , 2) lower power personal/portable devices, such as Wi-Fi network interface cards in laptops etc..
The fixed devices and some portable devices should access the WSDB to obtain the available spectrum and permitted transmission.
%
The fixed white space devices work as base stations and provides a backhaul for broadband client access.
These white space base stations (WBS) are the base stations in the IEEE 802.22 standard which are designed for wireless regional area networks.
As the fixed white space devices work with high transmission power, they are the major potential source of interference to the digital terrestrial TV receivers so that their transmission characteristics should be carefully decided by the WSDB.
%According to regulations, the calculation engine within the WSDB translates the information of incumbent services and technical characteristics and location of the white space devices into a list of allowed frequencies and associated transmit powers for the white space devices~\cite{ECC236}.
WSDB's decisions on the transmission power are different with respect to different regulations.
ECC and Ofcom~\cite{ECC186, ECMA392} emphasize on the protection of the critical points from the harmful interference, and don't regulate the transmission power limitation. 
As to decision on the transmission power per ECC, as WBSs work in underlay manner and coexist with digital terrestrial TV receivers, the aggregate generated interference caused on the critical points on each channel should not exceed a threshold.
FCC initially restricts the maximum transmission power of the fixed devices to be 1 Watt and now relaxes the limitation to 4 Watt.
FCC doesn't impose restriction on the aggregated interference caused on primary TV receivers, hence there should be a limit on the number of operating WBSs when the WBSs in the IEEE 802.22 network are dense.
%
The existence of WSDB makes it a natural choice to adopt centralized solution for the channel allocation problem, \ie the WSDB not only provide the WBSs with the available channels and corresponding maximum powers, but also the channels to work on at a given time.
But where to make the channel allocation decision is not regulated, thus we will also provide distributed solutions.

In this paper, we focus on the co-existence issue of the WBSs and look into the multiple channel allocation problem in TV white space with respect to ECC.
The main contributions of this work can be summarized as follows:
\begin{itemize} 
\item Complying with ECC rules, we devise both centralized and distributed schemes to make use of TVWS in cellular networks and improve the Shannon capacity on end users.
To our knowledge, this is the first effort made to utilize multiple TVWS channels under the ECC regulations. 
\item We formulate the channel allocation problem, where the the transmission powers are different as to different WBS and channels, into a congestion game, then derive the distributed algorithms which converge to Nash Equilibrium. 
\item We solve the centralized optimization problems and obtain the global optimal.
Simulation shows the distributed scheme achieves similar average capacity on end users but consumes considerably more energy.

\end{itemize}





%FCC issued a memorandum~\cite{FCC_2010_sedond_memorandumm,FCCdatabasae} in 2010, which removes the mandatory rigid sensing requirements, and prompts the usage of geolocations\footnote{Geolocation means both geographic location and terrain.}.
%FCC regulates a centralized database, which registers all the secondary users within one certain area, and decides on the available channels for them to use.
%The secondary users should access the database to obtain the list of available channels for the to use.
%The authors of ~\cite{SenseLess2011} validate this regulation and demonstrate the feasibility of only using geolocations and  propagation model.
%They adopt a central database which contains the geolocations of all TV stations.
%Then with sophisticated propagation model (Longley-Rice), the central database calculates the received signal strength index (RSSI) levels of TV UHF signals in a vast area.
%If RSSI on a channel is below a certain threshold on a location, TV service is regarded to be idle on that channel there and the secondary users there are allowed to use.
%The calculated results on channel availability is very close to the measurement results, which gives big impetus to the application of database mode in the exploration of TV white space.
%
%The FCC memorandum~\cite{FCC_2010_sedond_memorandumm,FCCdatabasae} and the work~\cite{SenseLess2011} initialize a new and easier way to utilize the TV white space, and the work~\cite{SenseLess2011} illustrates it is feasible to decide the RSSI level only with appropriate propagation model and geolocation.
%%Given the geolocation of the TV stations and secondary users, and appropriate propagation model, secondary users' maximum transmission power can be determined by the central entity according to the interference margin (maximum RSSI level from secondary users) on the TV receivers. 

%%ZZZZ modify the following paragraphs ZZZ
%In this paper, we investigate the efficient way to exploit the TV spectrum in a wireless regional area network which complies with IEEE 802.22 network.
%The secondary users are assumed to be cellular base stations and associated terminals, all of which work on TV white spectrum. 
%The base station is referred as WBS.
%%The corresponding secondary base stations are referred as white base stations (WBS). 
%Some cellular networks, \ie GSM or LTE network, work on licensed spectrum and emphasis on providing satisfactory services to their end terminals by choosing proper transmission channel and power. 
%As to cellular network working on TV white spectrum, they have to keep one eye on the primary users to make sure that TV service is not violated, which makes the problem of channel and power selection difficult.
%With the existence of central database, it is natural to utilize it as a central controller to assign channel and power usage for secondary users, but the secondary users may belong to different commercial groups and they may not contend with the assigned resource.
%%Besides, as the TV channels have different quality, \ie interference level, and permitted transmission power, it is difficult for the database to assign them to the 
%Hence, the spectrum sharing of the secondary users in IEEE 802.22 network should be decided in distributed manner and each secondary user takes care of its own interest, \ie to maximize its preferred utility.



The rest of the paper is organized as follows. 
We elucidate the system model in Section 2, afterwards related work is presented in Section 3. 
In Section 4, with respect to the regulation from ECC, we present the centralized optimization and game driven distributed scheme in terms of channel allocation.
In Section 5, with respect to FCC regulations, both centralized optimization and game driven distributed schemes are introduced.
Thereafter performance evaluation is presented in Section 6.
Finally, we conclude our work and point out directions of future research in Section 7.


\section{System Model and Problem Statement}
\label{SystemModel}
%According to the IEEE 802.22 standard, the primary systems considered in this chapter are digital TV (DTV) stations which use the TV spectrum legally. 
%TV stations provide service to passive TV receivers.
%The secondary users are IEEE 802.22 Wireless Regional Area Network base stations utilizing the TV spectrum with senseless mode~\cite{SenseLess2011}. 
%DTV's service should not be interfered by secondary systems. 

We consider an IEEE 802.22 compliant cellular network where the fixed white space devices work as base station and provide broadband access to their terminals.
The network is illustrated in Figure~\ref{sysmodel}.
We call the fixed white space devices which work as base stations as White space Base Stations (WBSs), they are located in one area which is surrounded by digital TV stations and receivers.
Several critical points are deployed in the vicinity of the digital TV receivers which are the most venerable to the interference caused by the white space devices.
WBSs work in underlay manner and coexist with the digital TV stations and receivers, the aggregate interference generated by WBSs should not exceed the threshold on each channel at each critical point.
%The aggregated interference on these critical points should below a certain threshold on all of the TVWS channels.
%
As to the WBSs, the out-of-band emission is regard as trivial, therefore, we only consider co-channel interference among the WBSs.
To simplify the analysis, we assume that each digital TV station as well as each WBS utilizes exactly one channel.\footnote{The assumption that one WBS only utilizes one channel is for convenience of analysis. In reality multiple channel usage (channel bonding) is requirement as one single TV channel's bandwidth is 6 MHz which is not adequate for a WBS to fulfill terminal demands. 
%We will relax this single channel usage assumption without hammering our scheme in the end of section \ref{sec_CA}.
}
%

As to the notations, the set of critical points is denoted as $\mathcal{K}$ and the set of WBSs is denoted as $\mathcal{N}$ where $|\mathcal{N}|=N$. 
The TVWS spectrum bands are denoted as set $\mathbb{C}$.
We represent the usage of channel for WBS $i$ with a binary vector $X_i^{|\mathbb{C}|\times 1}=\{x_i^1,\cdots, x_i^k,\cdots, x_i^{|\mathbb{C}|}\}\in \{0,1\}^{|\mathbb{C}|}$, where $k\in \mathbb{C}$ and binary variable $x_i^k$ denotes whether channel $k$ is used by user $i$. 
All the WBSs work with the channels approved by the WSDB, they operate with a channel from the approved ones after choosing it, thus we omit the time index in the channel usage.
As each node can only uses one channel, for $X_i$, there is $\sum_{k=1}^{|\mathbb{C}|}x_i^k=1$. 
The transmission power of WBS $i$ on channel $c$ is $P_i^c$.
$c(i)$ denotes the channel used by a WBS $i\in \mathcal{N}$. 

\begin{figure}[h!]
  \centering
  \includegraphics[width=0.9\linewidth]{systemmodel_working.pdf}
  \caption{System model: WBS cells and digital TV (DTV) systems}
\label{sysmodel}
\end{figure}
%In the rest of the chapter, we use WBS and secondary base station interchangeably. 
%There are interference measurement equipments deployed on the contours of TV service areas (as bold rectangles in Fig.~\ref{sysmodel}), which represent the worst located TV receivers in the TV service areas. 
%For these interference measurement devices, an interference threshold should not be violated by the noise generated by the secondary users.
%The deployment of the interference measurement devices is decided by the TV operators, which are usually along the contour of the area where TV receivers reside.
%Thus, the locations of interference measurement devices vary according to the concrete location, geographic terrain and possible deployment of secondary networks. 
%%For simplicity, we assume there is only one contour deployed for one TV area. % \todo{is contour 'clear' now?}.
%WBSs are deemed to be static.
%We assume the secondary base stations are not under the same operators, thus there is no scheduling mechanism available among WBSs.


For a terminal $m$ which is associated to WBS $i$, the attenuation between WBS $i$ and $m$ is denoted as $h_{im}$.
For the attenuation, we only take path loss and shadowing into account in the following.
The path loss is dependent on the distance between the corresponding equipment, e.g. $h_{im}=K \cdot d_{im}^{-\alpha}$, where $\alpha$ is the path loss exponent, $d_{im}$ is the distance between $i$ and $m$, while $K$ is a constant which models the reference loss over a single unit of distance.
Shadowing without fading is considered in our model.
$z_{im}$ models the zero-mean log-normally distributed shadow fading between $i$ and $m$, with the standard deviation $\sigma_{\text{SH}}$.
$N_0$ denotes the thermal noise power.
%
%The sum of all disturbing radio frequency effects (including interference) on terminal $m$ (we assume the working channel is $c$) is as following,
%\begin{equation}
%\label{interference}
%\begin{aligned}
%f_m^c=\sum_{\bar{i}} (P_{j}^c \cdot h_{jm} \cdot z_{jm}) +  N_0, \quad \quad j\in \mathcal{N}\setminus i, c(j) = c
%\end{aligned}
%\end{equation}
%where $P_{j}^c$ denotes the transmission power of interfering WBS $j$.
%Note that $z$ is dependent on the individual transmitter/receiver pair, but we omit the subscripts for simplicity. 
The SINR at end terminal $m$ is,
\begin{equation}
\label{SINR}
\begin{aligned}
\gamma_{m}  & = \frac{P_{i}^c \cdot h_{im}\cdot z_{im}} {f_m^c} & = \frac{P_{i}^c \cdot h_{im}\cdot z_{im}} {\sum (P_{j}^c \cdot h_{jm} \cdot z_{jm}) +  N_0}\\
					& &, \quad \quad j\in \mathcal{N}\setminus i, c(j) = c
\end{aligned}
\end{equation}
where $P_{j}^c$ denotes the transmission power of interfering WBS $j$.



%\subsection{Problem Statement}
In our model, we only assume the WBSs, which work with high transmission power, as the potential interfering unlicensed devices to the DTV service, meanwhile, WBSs are interested in providing broadband access to their associated terminals.
Our goal in this paper is to assign TVWS channel to each WBS so as to improve the signal to noise and interference ratio (SINR) of their associated end terminals, meanwhile complying with the ECC regulations.
The channels in $\mathbb{C}$ are assumed to be identical in terms of attenuation and shadowing on the same path.
%As to performance metric for the QoS provisioning, we choose the signal to noise and interference ratio (SINR) on the terminals.
A WBS's utility is a function of the SINR on all its end terminals, \ie the average SINR at all its terminals.


\subsection{Problem Formulation}
\label{problemProposed}
Our goal is to minimize the sum of inverted SINR the WBSs provide to their end terminals $\sum_{i\in \mathcal{N}}\frac{1}{\gamma_{i}}$ .
In order to ensure the fairness among WBSs,  we minimize the sum of inverted quasiSINR instead of maximizing the sum of quasiSINR of all WBSs.
%
As to WBS's utility on terminals' SINR, it is not appropriate only to choose one terminal, as done in~\cite{spectrum_sharing_tvspace_2012}, or even multiple fixed terminals to represent the all the terminals in the same cell, because their locations could diverge greatly with the locations of the other terminals.
Thus, we propose a metric \textit{QuasiSINR} to represent WBS's performance in terms of SINR on its end terminals, which is independent on the actual locations of end terminals.


With an auxiliary circle centered at the discussed WBS, which is shown as dashed circle in Figure~\ref{quasiSINRfigure}, QuasiSINR is the ratio between the power of signal of interest on the circle and the summation of the strongest power from the interfering WBSs on the auxiliary circle.


\begin{figure}[h!]
  \centering
  \includegraphics[width=0.6\linewidth]{quasiSINR2_2.pdf}
  \caption{Assuming the radius of the auxiliary circle is $\delta$ and all the WBSs work on the a channel, then QuasiSINR is a quotient where the divided is the WBS $i$'s power on the auxiliary circle (\ie the green pot), and the divisor is the summation of the interfering power on the red pots.}
\label{quasiSINRfigure}
\end{figure}

The quasiSINR of WBS $i$ is denoted as $\gamma_{i}$, 



\begin{equation}
\label{quasiSINR}
\begin{aligned}
 \gamma_{i} = &\frac{P_{i}^c \cdot h_{i\rightarrow \text{i's}\hspace{0.2em} \text{auxiliary circle}}\cdot z_{i\rightarrow \text{i's}\hspace{0.2em}\text{auxiliary circle}}} {\sum\limits_{\tiny\substack{j\neq i, j\in \mathcal{N}\\c(j)=c(i)}} (P_j^c \cdot h_{j\rightarrow \text{i's}\hspace{0.2em} \text{auxiliary circle}} \cdot z_{j\rightarrow \text{i's}\hspace{0.2em} \text{auxiliary circle}}) + N_0}\\
 = & \frac{P_{i}^c \cdot \delta^{-\alpha}\cdot z_{i\rightarrow \text{auxiliary circle}}} {\sum\limits_{\tiny\substack{j\neq i, j\in \mathcal{N}\\c(j)=c(i)}} (P_j^c \cdot (d_{ji}-\delta)^{-\alpha} \cdot z_{j\rightarrow \text{auxiliary circle}}) + N_0}
\end{aligned}
\end{equation}

In the following paper, when we talk about the channel and power allocation with respect to WBS, the notation $h_{ij}$ denotes the attenuation between WBS $i$ to the auxiliary circle of WBS $j$.
$h_{i}$ denotes the attenuation between WBS $i$ to its own auxiliary circle.
Then $\gamma_i$ becomes,
\begin{equation}
\label{quasiSINR_2}
\begin{aligned}
 \gamma_{i} = 
  \frac{P_{i}^c \cdot h_i\cdot z_i} {\sum\limits_{\tiny\substack{j\neq i, j\in \mathcal{N}\\c(j)=c(i)}} (P_j^c \cdot h_{ji} \cdot z_{ji}) + N_0}
\end{aligned}
\end{equation}
The abbreviations and notations used in this paper are found in Table~\ref{tab1}.
%
%With auxiliary circle, the decision made by WBSs is independent on the distribution of the end terminals.
The radius of the auxiliary circle $\delta$ can be adapted to foster better service to the terminals in certain area, \ie a larger radius $\delta$ will take care of the SINR on the terminals reside far away and vice visa.


\begin{table}[h]
\caption{Notations}
\label{tab1}
\centering
\begin{tabular}{l p{5.5cm}}
\toprule
Abbr. & Description \\
Symbol & \\
\midrule
TVWS & TV white spaces\\
WSDB & white space database\\
WBS & white space base stations\\
$\gamma$ & QuasiSINR\\
$f_{ji}^c$  & The co-channel interference caused by WBS $j$ on the auxiliary circle of WBS $i$, $c$ is the working channel for both\\
$f_i^c$ & The sum of interference caused on the auxiliary circle of WBS $i$ \\
$p_i^c$		& The Tx power of WBS $i$ on channel $c$ on channel $c$\\
$P_i^c$		& The maximal permitted Tx power of WBS $i$ on channel $c$ (ECC solution)\\
$P_\mu, P_{\mathtt{op}}$		& The minimum and maximal permitted Tx power of WBS $i$ (FCC solution)\\
$h_{ij}$ & The attenuation between WBS $i$ to the auxiliary circle of WBS $j$.\\
$h_{i}$ & The attenuation between WBS $i$ to its won auxiliary circle.\\
$z_{ij}$ & The shadowing from WBS $i$ to the auxiliary circle of WBS $j$.\\
$z_{i}$ & in Section~\ref{whitecat}, the shadowing from WBS $i$ to its own auxiliary circle.\\
$\alpha_{ij}^k,\beta_{ij}^k$ & Binary auxiliary variables in the\\
$y_i, z_i$ & in Section~\ref{WhiteSussa}, optimization parameters.\\
$\text{cp}$ & Critical point\\
\bottomrule
\end{tabular}
\end{table}



In our model, WBSs access the WSDB and obtain the transmission parameters, \ie working channel, transmission power of the other WBSs, the attenuation characteristics between itself and all the other WBSs, and vice visa. 
WBSs calculate their QuasiSINRs with these information respectively.






\section{Related Works}
\label{decomposition_relatedwork}



%In related works, the protection on primary users is taken care in the same time when channel and power selection are conducted.
%But according to the current regulations and standards, there exist no communication means between the secondary users and the primary users.
%Besides, when assuming such communication media is available and preventing primary users from being interfered during secondary users' power and channel allocation, the communication overhead between primary users and second users is considerable.

To exploit TVWS, authors in \cite{DySpAN10MeasuringWhitespaceCapacity, HessarTMC15, Deshmukh2015, Achtzehn12} have proposed different approaches for assessment of TVWS capacity under FCC and ECC regulations respectively.
Hessar et al.~\cite{ReAlloTVWS14DySPAN} aim to maximize the Shannon capacity of the network which complies with FCC rules.
The solution seeks the trade-off between the benefits brought in by wide band and co-channel interference which is resulted on the wide band.
Centralized schemes are proposed to increase the throughput of the secondary work, but they only focus on the capacities on the location of the secondary base stations.
% This statement is wrong: But this scheme doesn't restrict the number of TVWSs and doesn't consider the harmful interference caused on the digital terrestrial TV receivers.
Yang et al.~\cite{yang2013WiFiWSTVCapacity} and Gopal et al.~\cite{gopalTCCN16} follow the rules of FCC, and propose throughput maximization of a CSMA/CA based WiFi like network in TVWS under aggregate interference.
Ying et al.~\cite{Ying2018DySPAN} look for the maximal set of white space devices which could access the spectrum, a heuristic scheme is proposed based on maximum weighted independent set in the graph which representing the secondary network.
This work doesn't consider the aggregate interference caused to the digital TV receivers.
Conflict graph is used in many works folllowing FCC rules. Ying et al,~\cite{conflictGraphTWC2018} increases the percentage of nodes served and the number of assigned channels. A Wi-Fi like secondary network is formulated into a conflict graph with consideration on pairwise interference and channel contiguity. 


%TODO-to modify
Omidvar et al. ~\cite{pimrc_2012} use potential game to propose a distributed joint power and channel allocation in cognitive radio network.
Although the scheme is not tailored for TVWS, protection on the primary users are also considered.
Potential game is adopted in work~\cite{Elias17} to mitigate the adjacent interference meanwhile bonding multiple channels, but the white space devices adopt fixed and identical transmission power.
In ~\cite{spectrum_sharing_tvspace_2012}, Chen et al. formulate the channel allocation problem in TV white space into a potential game where individual WBS's utility is to maximize the capacity of one static terminal.
%TODO: read the commented part.
%The execution of this scheme is formulated into an exact potential game. 
%For each base station, after several rounds of best responses in terms of channel and power level, Nash equilibrium is achieved.
%There are some flaws hindering the application of this scheme.
%Firstly, the paper doesn't provide means for base stations to obtain the needed information which is needed to calculate the utility function.
%Secondly, it is not clear how to calculate the punishment in the utility function, which indicates whether and how much the interference threshold on primary users is violated.
%Thirdly, the convergence speed of the scheme is not given, in fact, as the problem is formulated into a potential game, converge speed or the number of updates before convergence is a theoretic problem which is still unsolved.
%Last but not least, as the utility function and the potential in the game are designed as the sum of received and introduced interference, the desired signal power and the punishment, the minimization of this \textit{sum} does indicate meaningful  performance metrics, \ie SINR on terminals, or the total transmission power consumption.



With the doctrine of not interfering the primary TV services and regulations imposed by institutes, the problem we solve in this paper is different from the channel allocation problems discussed in the domain of cognitive radio networks.
But we think a brief review in terms of the channel allocation techniques in CRN is necessary.
%
%Channel assignment problem can be converted into colouring problem thus is NP hard~\cite{Hyacinth}. 
%Authors of~\cite{Ko_DistributedCA} propose best response to improve the transmitters' utilities, but best response process can only converge when the transmission powers are identical and path loss is symmetric.
With identical transmission power and symetric path attenuation, Nie et al.~\cite{CApotentialLearning_05dyspan} formulate channel assignment problem in ad-hoc cognitive radio network into a potential game which leads to pure NE, the authors~\cite{CA_Felegyhazi_07infocom, Wu_GOP_CA_08infocom} propose algorithms which converge to pure Nash equilibrium (NE) and strongly dominate strategy equilibrium respectively. 
Simulated annealing is applied to mitigate co-channel interferences in~\cite{SA_CA_TVWS_2012crowncom}.
For the same purpose, no-regret learning~\cite{qlearning_huang, hart00correlatedeq} is exploit to optimize the choice on channel.
In~\cite{hart00correlatedeq}, each WBS maps the probability of choosing each strategy to a certain proportion of the regret which the WBS may have if it doesn't choose that strategy, and the WBS choose the strategy with the biggest probability.  
WBSs update such mapping dynamically and this approach converges to correlated equilibrium. 

%As to our knowledge, there is no work dealing with channel allocation problem where transmission power is different.
%XXX So here we have the same problem as above: Only for the first two papers you say explicitly what problems come up with them - for the rest you simply mention them and their approach. Here we really need a much more clearer formulation of the shortcomings - why are they failing to be a solution to our problem ? Furthermore, there seem to be quite a lot of work in game theory and power control in secondary networks - how do they relate to our work? XXX

%Channel allocation facilitates CRN to improve throughput~\cite{channelAllocation_throughput_12wcnc}, or cooperatively relay~\cite{channelAllocation_relay_2010ICASSP} and so on.
%This thesis emphasises on co-channel interference mitigation with distributed channel allocation. 



%In this work we concern the best possible quality-of-service provisioning by allocating channels on which fixed transmission power is used.
%In this paper we try to improve the SINR on secondary end terminals through WBSs' power-channel strategy. To facilitate analysis and proposition of solution, we propose a metric \textit{QuasiSINR} for each secondary base station to represent the SINR of the terminals in the coverage of that base station.


%\todo{todo:}
%\begin{itemize}
%\item what is the reason to design a distributed scheme for this problem.
%\begin{itemize}
%\item complexity (to w)
%\item overhead
%\item WBSs belong to different commercial organization (we actually propose a spectrum sharing paradigm)
%\end{itemize}
%
%\item argument 1: why is the utility used in the algorithm, in other words, what is the reason to chose it and what is the relation between it and the SINR on end users.
%
%\end{itemize}

\section{Channel and Power Allocation Scheme complying with ECC Rules}

In this section we will discuss the channel and power allocation problem complying with the rules regulated by ECC.
Given all the other WBSs' selection on channel and transmission power, a WBS is interested in choosing the channel which brings it the best performance, \ie the data rate of its end users.
A WBS prefers to choose the channel which experiences the minimum interference, while simultaneously the transmission power can be set as high as possible, so as to obtain better SINR at associated terminals and and generally increase coverage. \cite{wuinfocom09, HoangPowerChannel2010}. 
Nevertheless, high transmission power causes significant co-channel interference to other secondary users operating on the same channel. 
Hence, a secondary cell has to balance its transmission power and the caused interference on other cells while simultaneously choosing a working channel that decreases the interference its terminals are exposed to.
In the following subsections, we firstly present the decision on the maximum permitted transmission power on each channel for each WBS, then present how do the WBSs make use of these powers in both centralized and distributed manner respectively.
\subsection{The Maximum Permitted Transmission Power}
\label{powermap}

We adopt the interference model and the optimization methodology from the work of \cite{multipleIntf_pimrc11} to plan the maximum transmission power on each channel for WBSs.
%
For WBS $i\in \mathcal{N}$, the maximum transmission power allowed on channel $c\in \mathbb{C}$ is denoted as $P_i^c$. 
As to each channel $c$, the generated interference on each interference measuring device should be within a predefined interference margin $I$.
The interference margin in a slow fading environment is decided according to~\cite{aggregate_interference_shadow_fading_2010}.
To address this fairness issue, we maximize the sum of the logarithmic value of every WBS's transmission power, and formulate the problem into a convex optimization problem.
	\begin{equation}
		\label{cvx}
		\begin{aligned}
		& {\text{Maximize}}
		& & \sum_{i\in \mathcal{N}} \log P^c_i \\
		& \text{subject to}
		& & \sum_{i\in \mathcal{N}} (P^c_i \cdot h_{i,\text{cp}}\cdot z) < I, \forall\text{cp} \in \mathcal{K}
		\end{aligned}
	\end{equation}
This optimization is conducted in WSDB for each channel $c\in \mathbb{C}$, the obtained power-channel map will be used by the WBSs to decide the operation channel afterwards.
The interference threshold on the critical points will not be exceeded even when all the WBSs work on one same channel.

%there are multiple channels available and WBSs are free to choose their preferred channels, the aggregate interference on one channel will be smaller than that when all WBSs work on that channel. 
%When the WBSs work on different channels, there will be a margin for network dynamics such as new WBS starting to work or increased interference on TV contour due to the variance of broadcast path condition. 
%XXX As mentioned above, it is not clear which impact the relationship between maximum transmit power planning and channel selection has XXX.


\subsection{Centralized Optimization for Channel Allocation}
\label{03_centralized_ca}

%We formulate the channel allocation problem into a binary quadratic programming problem.  
%Let $X_i = \{x_i^1,\cdots, x_i^k,\cdots, x_i^{|\mathbb{C}|}\}$ denote the vector of channel usage, there is $|X_i| = |\mathbb{C}|$ and binary element $x_i^k$ represent whether WBS $i$ occupies channel $k$.
Given two WBSs $i$ and $j$, the co-channel interfering relationship is decided as,
\begin{equation}
\begin{split}
X_i^TX_j = \sum\limits_{k=1}^{|\mathbb{C}|}x_i^k\cdot x_j^k = 
\left\{ \begin{array}{ll}
1 & \mbox{if $c(i)=c(j)$} \\
0 & \mbox{if $c(i)\neq c(j)$} 
\end{array}
\right.
\end{split}
\end{equation}

%$c(i)$ means the working channel which is chosen by WBS $i$.
The transmission power levels on all channels for WBS $i$ are denoted by a constant vector $\vec{P}_i = \{P_i^1,\cdots, P_i^k,\cdots, P_i^{|\mathbb{C}|}\}$. 
The transmission power adopted by WBS $i$ is $\vec{P}_i^TX_i = \sum\limits_{k=1}^{|\mathbb{C}|}P_{i}^k\cdot x_i^k$.


The problem in Section~\ref{problemProposed} is formulated as a general purpose nonlinear optimization,
	\begin{equation}
\label{03_centralized_ca_Optimization}
		\begin{aligned}
		& \underset{}{\text{minimize}}
		& & \sum\limits^{n}_{i=1} \frac{\sum\limits_{j\in\mathcal{N}, j\neq i}\vec{P}_j^TX_j(X_j^TX_i)h_{ji}z_{ji} + N_0}{\vec{P}^TX_ih_iz_i}\\
		& \text{subject to}
		& & \sum\limits_{k=1}^{|\mathbb{C}|}x_i^k=w, x_i^k\in X_i\in \{0,1\}^{|\mathbb{C}|}\\
		\end{aligned}
	\end{equation}
where $w$ is the number of channels which are used by WBSs. 
Optimization formulation \ref{03_centralized_ca_Optimization} is non-linear with binary variables, but it can be transformed into a quadratic programming problem as follows,
	\begin{equation}
\label{QLP_2}
			\begin{aligned}
			\underset{}{\text{minimize}}
			& \sum\limits^{n}_{i=1} ( \sum\limits_{j\in\mathcal{N}, j\neq i}\sum\limits_{k\in\mathbb{C}} \frac{P_j^k\cdot h_{ji}\cdot z_{ji}}{P_i^k\cdot h_i\cdot z_i}\cdot  x_j^k\cdot x_i^k  \\
			& + \sum\limits_{k\in\mathbb{C}} \frac{N_0}{P_i^k\cdot h_i\cdot z_i}\cdot x_i^k)\\
			\text{subject to} 
			& \sum\limits_{k=1}^{|\mathbb{C}|}x_i^k=w, x_i^k\in X_i\in \{0,1\}^{|\mathbb{C}|}\\
			\end{aligned}
		\end{equation}

The reformulation is available in Appendix in our previous work \cite{Li2012DistributedTS}.
We use GUROBI~\cite{gurobi} which is a state of art non-linear problem solver to solve the problem, which employs Branch-And-Reduce method to get the global optimum for the problem. % We use the results obtained by solving this QLP problem as one reference in the coming section. 
%The result of this centralized channel assignment will be evaluated in the simulation section with other schemes. 



\subsection{Distributed White Space Channel Allocation (WhiteCat)}
\label{whitecat}
In this section a distributed scheme for WBSs to allocate multiple channels meanwhile complying with ECC regulation is proposed, which is named as \underline{white} space \underline{c}hannel \underline{a}llocation \underline{t}echnology (WhiteCat). 
We regard a WBS working with a certain channel as a \textit{logic} WBS, in other words, a WBSs operating on multiple channels is seen as multiple co=location WBSs which operate on s single channel. 
In Section~\ref{whitecat}, the WBSs are logic WBSs.
\subsubsection{Algorithm and Protocol}

WhiteCat adopts the best response process, where each WBS (referred as $i$) chooses the channel which brings the largest utility $u_i$, as shown in Formula~\ref{utility}, as the response of other WBSs' choices on channels.
%and the sum of all WBSs' utilities is minimized after finite times of updates even the interaction between WBSs are asymmetric. The utility is as follows,
WhiteCat is depicted by Algorithm \ref{whitecatalgo}.

\begin{equation}
\label{utility}
u_i =\dfrac{\sum\limits_{\tiny\substack{j\in \mathcal{N}, j\ne i,\\ c(\sigma_j)=c(\sigma_i)}}f_{ji}}{2\cdot \tilde{P_i}} + \dfrac{1}{2}\sum_{\tiny\substack{j\in \mathcal{N}, j\ne i,\\ c(\sigma_j)=c(\sigma_i)}}\dfrac{f_{ij}}{\tilde{P_j}} + \sum_{\tiny\substack{\mathcal{S}:i,j\in \mathcal{S},\\ c(\sigma_j)=c(\sigma_i)}}\dfrac{N_0}{C\cdot \tilde{P_i}}\\
\end{equation}

where $\tilde{P_i} = P_i\cdot h_i\cdot z_i$, similarly $\tilde{P_j} = P_j\cdot h_j\cdot z_j$.
Overlooking the constant coefficient 2, the first item of $u_i$ is the inverted QuasiSINR of station $i$. 
To minimize the first item, WBS $i$ needs to choose a channel either permits higher transmission power or experiences less interference, whereas the higher power increases the second item which is a part of inverted QuasiSINR of other co-channel WBSs. 
Hence, the cost function presents a reasonable compromise between the QuasiSINR of one WBS and others.

When WBS only emphasizes on its own utility (e.g. the first part of Formula~\ref{utility}), the best response process doesn't always converge and we have following theorem:
\begin{theorem}
\label{noconvergence}
\emph{With non-identical transmission power, if every WBS updates its channel based on Algorithm \ref{whitecatalgo} with utility based on its own interests, \ie the first part of Formula~\ref{utility}, the process doesn't always converge.}
\end{theorem}
The proof is in Appendix 1 in~\cite{Li2012DistributedTS}.


\begin{algorithm}[h]
\caption{Spectrum selection by WBS $i$}          % give the algorithm a caption
\label{whitecatalgo} 
\DontPrintSemicolon
\SetAlgoLined
\KwIn{the distance, path lose and shadowing parameter between WBS $i$ to WBS $j\in \mathcal{N}\setminus i$;\\  radius of auxiliary circle, noise $N_0$, total number of WBSs $N$;\\ for $j\in \mathcal{N}\setminus i$, the maximal transmission power $P_j^c, c\in \mathbb{C}$ and the working channel $c(j)$.
}
%\KwOut{a channel }

	\For{$c\in \mathbb{C}\setminus c(i)$}{
	 calculate $u_i(c)$ based on Formula \ref{utility}
	 \eIf{$u_i(c)<u_i(c(i))$}{
	 	$c(i)\leftarrow c$
	 }
	 {keep $c(i)$ unchanged}
	}
	
Notify database of its channel usage, which further notifies the other WBSs

\end{algorithm}

%Some parameters needed to calculate the utility are identical for all WBSs, such as quasi distance $e$, the total number of WBSs $N$, number of channels $C$, attenuation factor $\alpha$, standard deviation $\sigma_{WBS}$ in flat shadowing and noise $N_0$, albeit 

%Similar with~\cite{SenseLess2011}, we let every WBS store the location information and maximal power map of all other WBSs, \ie $P_i^c, i\in\mathcal{N}, c\in\mathbb{C}$, and each WBS retrieves information about channel usage of other WBSs from centralized base station.

WBSs calculate $u_i$ with the information retrieved from WSDB.
After executing Algorithm \ref{whitecatalgo}, WBSs report to WSDB about their chosen channels if they are updated.
As the locations of WBSs and TV stations, along with the transmission channel and power of TV stations are usually static (entries of TV station change averagely once in 2 days\cite{SenseLess2011}), except for the channel usage in the network, the change of the other data stored in WBS is infrequent. 
We refer \cite{CApotentialLearning_05dyspan} to decide WBSs' sequence to update their channels.
\cite{CApotentialLearning_05dyspan} proposes a method which is akin to the random access mechanism of CSMA/DA, where the access for broadcast medium is changed to get access to the centralized center to retrieve the current channel usage and update its new channel. 
All WBSs are able to access the database in one round (with random or predetermined sequence). 
%As WBSs are connected with database, the control messages needed to decide the sequence will not become a burden. 
Update of channels can happen in several scenarios, \ie in boot phase, or when the SINR on end users falling below a threshold, or a fixed time duration coming to end, or new WBSs joining in the network. 


\subsubsection{Analysis in Game Theoretical Framework}
\label{game}
In this section, We give the proof on whiteCat's convergence in the framework of congestion game theory.
Formulating a spectrum sharing problem into a congestion game and the concept of \textit{virtual resources} are firstly proposed in \cite{allerton08_liu}.
This work reversely engineers the distributed channel allocation schemes proposed in \cite{babadi_08, Ko_DistributedCA}, \ie unifies the algorithms with congestion game.
But the problem analysed in~\cite{allerton08_liu} assume the transmission power is identical, which is a major difference from the channel allocation problem discussed here. 

%\subsubsection{Congestion Game}
%A congestion game \cite{Rosenthal}\cite{Voecking06congestiongames} can be expressed by a tuple $\lambda=(\mathcal{N},\mathcal{R},(\sum_i)_{i \in \mathcal{N}},(g_r)_{r\in \mathcal{R}})$, where $\mathcal{N}=\left\{1,\ldots,N\right\}$ denotes the set of players (each each is labeled with a unique index number), $\mathcal{R}=\left\{1,\ldots,m\right\}$ the set of resources, $\Sigma_{i\in\mathcal{N}} \subseteq 2^{\mathcal{R}}$ is the strategy space of player $i$. Under strategy profile $\sigma=(\sigma_1,\sigma_2,\cdots \sigma_N)$, player $i$ chooses strategy $\sigma_i\in \Sigma_i$, and the total number of users using resource $r$ is $n_r(\sigma)=|\{i\mid r\in \sigma_i\}|$. The cost $g_r: \mathbb{N}\rightarrow \mathbb{Z}$ is a function of the number of users for resource $r$, $g_r^i=\sum_{r\in \sigma_i} g_r(n_r(\sigma))$. In our paper, $g_r^i$ is referred as \textit{congestion} and is Monotonic.
%
%Rosenthal's potential function $\phi:\sigma_1\times\sigma_2\times\cdots\times\sigma_n\rightarrow Z$ is defined as:
%\begin{equation}
%\label{4}
%\begin{split}
%G(\sigma) 
%& =\sum\limits^{}_{r\in \mathcal{R}} \sum\limits^{n_r(\sigma)}_{i=1} g_r(i)\\
%& =\sum\limits_{i\in \mathcal{N}} \sum\limits^{}_{r\in \sigma_i} g_r(n_r^i(\sigma))\\
%\end{split}
%\end{equation}
%$n_r^i(\sigma)$ means the number of players using resource $r$ and \textit{their indices are smaller than or equal to $i$}. Note that the potential is \textit{not} the sum of congestions experienced by every user. The change of the potential caused by one player's unilateral move from $\sigma$ to $\sigma'$ is equivalent to the change of gain (or loss) of that player.
%\begin{equation}
%\label{5}
%\varDelta G(\sigma_i \rightarrow \sigma_i') = g^i(\sigma_i',\sigma_{-i}) - g^i(\sigma_i,\sigma_{-i})
%\end{equation}
%$\sigma_{-i}$ is the strategy profile for all players except for $i$.
%As every congestion game is a potential game, and the total potential is finite, thus the number of improvements is upper-bounded by $2\cdot\sum\limits^{}_{r\in \mathcal{R}} \sum\limits^{n_r(\sigma)}_{i=1} g_r(i)$ \cite{Voecking06congestiongames}.
%
%
%We have introduced congestion game in Chapter~\ref{background}, thus we only recap the essence of congestion game here.
In congestion game, each player acts selfishly and aims at choosing strategy $\sigma_i\in \Sigma_i$ to minimize their individual cost.
The gain (loss) caused by any player's unilateral move is exactly the same as the gain (loss) in the potential, which may be viewed as a global objective function.
For problems where the potential of the problem is the same with the summation of the cost of all users, the cost function can be used as a utility function directly.
This equivalence doesn't exist in our problem, but by carefully choosing the cost function for players, we can make sure that the change of individuals' cost is in the same direction with that of the global utility.



\subsubsection{From WhiteCat to Congestion Game}
\label{gameforproblem}
We utilize the conception of virtual resource which is firstly introduced in \cite{allerton08_liu}. 
Virtual resource is a triplet $\{i, j, c\}$, where $i,j$ are two WBSs and $c\in \mathbb{C}$ is one channel.
This piece of resource is regarded used by $i$ when both $i$ and $j$ use channel $c$, otherwise, $\{i, j, c\}$ is not used by any WBS.

In the following, we list the element of the congestion game which emulates Algorithm~\ref{whitecatalgo}.
In this section, player and base station are used interchangeably.

\begin{itemize}
\item Player $i$' strategy space is $\Sigma_i=\{(i,j,c), j\in \mathcal{N}, j\ne i, c(\sigma_j)=c, c=1,2,\cdots,N\}$, and $i$ has $C$ admissible strategies, one strategy related with channel $c\in\mathbb{C}$ is described by the set of virtual resources it uses: $\sigma_i=\{(i,j,c), j\in \mathcal{N}, j\ne i, c(\sigma_j)=c\}$, note that virtual resource $(i,j,c)\neq(j,i,c)$.

\item Under the strategy profile $\sigma=(\sigma_1, \sigma_2, \cdots \sigma_N)$, player $i$ obtains a total cost of 

	\begin{equation}
\label{6}
		\begin{split}
		g^i(\sigma)= \sum\limits^{}_{\tiny\substack{j\in \mathcal{N},j\neq i,\\ c=c(\sigma_i)=c(\sigma_j)}} &(g_{(i,j,c)}(n_{(i,j,c)}(\sigma))\\&+g_{(j,i,c)}(n_{(j,i,c)}(\sigma))
		\end{split}
		\end{equation}
\end{itemize}

The transmission power over all channels of player $i$ is $\{p_i^1, p_i^2,\cdots, p_i^{|\mathbb{C}|}\}$.
%According to our system model, interfere xxxxxx
%Path loss is assumed reciprocal: $h_{ij}=h_{ji}$, but nor is the flat fading $z$. To keep the formula clear in the following part, we denote $\tilde{f_{ij}}= P_i\cdot h_{ij}\cdot z$, $\tilde{f_{ji}}= P_j\cdot h_{ij}\cdot z$, $\tilde{P_i}=h_{iQ}$ for $i\in \mathcal{N}$, where $h_{ji}=h_{ij}=(d_{ji}-e)^{-\alpha}, h_{ii}=h_{jj}=e^{-\alpha}$, $d_{ji}$ is the distance between base station $i$ and $j$, and $\delta$ is the quasi distance introduced in section \ref{SystemModel}. $N_0$ is noise which is identical for any channel and any WBS. 
We define the cost function for virtual recourses $(i,j,c)$ as follows,
\begin{equation}
\label{costfuc4resrc}
\begin{split}
g_{(i,j,c)}(k) = 
\left\{ \begin{array}{ll}
%\tilde{f_{ji}}/(2\tilde{P_i}) + f_{ij}/(2\tilde{P_j}) + N_0/(C\cdot \tilde{P_i})\\\\
%=\dfrac{P_j\cdot h_{ji}\cdot z/2 + N_0/C}{P_i\cdot h_{iQ}\cdot z} +\dfrac{P_i\cdot h_{ij}\cdot z/2}{P_j\cdot h_{jQ}\cdot z} & \mbox{if $k=2$} \\
\dfrac{f_{ji}}{2\tilde{P_i}} + \dfrac{f_{ij}}{2\tilde{P_j}} + \dfrac{C\cdot N_0}{N\cdot \tilde{P_i}} & \mbox{if $k=2$} \\
%=\dfrac{P_j\cdot h_{ji}\cdot z/2 + N_0/C}{P_i\cdot h_{iQ}\cdot z} +\dfrac{P_i\cdot h_{ij}\cdot z/2}{P_j\cdot h_{jQ}\cdot z} & \mbox{if $k=2$} \\
0 & \mbox{otherwise}
\end{array}
\right.
\end{split}
\end{equation}


As resource $(i,j,c)$ only lies in the strategy space of player $i$ and $j$, thus can only be accessed by this two players.
More specifically, according to Formula~\ref{costfuc4resrc}, the cost of resource $(i,j,c)$ is only decided by the number of players using it, which is either 0 or 2.
At the first glance, this is a player specific congestion game, as $g_{(i,j,c)}$ is decided by the relevant players' transmission power and inference.
But actually the resource ${(i,j,c)}$ excludes the players except for $i$ and $j$ from using it, thus the cost happened on this resource is only dependant on how many of players from the set $\{i, j\}$ to use it.
Hence, the cost is a function of the number of players using the resource, and this is a canonical congestion game.


Now we substitute Formula \ref{costfuc4resrc} to Formula \ref{6}, the total cost for user $i$ under strategy profile $\sigma$ . 

\begin{strip}
\begin{equation}
\label{cost1player}
\begin{split}
g^i(\sigma)
 &= \sum_{\substack{j\in \mathcal{N}\setminus i,\\ c=c(\sigma_j)=c(\sigma_i)}} (g_{(i,j,c)}(2) + g_{(j,i,c)}(2))
= \sum_{\substack{j\in \mathcal{N}\setminus i,\\ c(\sigma_j)=c(\sigma_i)}}(\dfrac{f_{ji}}{\tilde{P_i}} + \dfrac{f_{ij}}{\tilde{P_j}}+ \dfrac{C\cdot N_0}{N}(\dfrac{1}{\tilde P_i}+\dfrac{1}{\tilde P_j})) \\
& = \dfrac{\sum\limits_{\substack{j\in \mathcal{N}\setminus i,\\ c(\sigma_j)=c(\sigma_i)}}f_{ji}}{ \tilde{P_i}} + \sum_{\substack{j\in \mathcal{N}\setminus i, \\c(\sigma_j)=c(\sigma_i)}}\dfrac{f_{ij}}{\tilde{P_j}} + \dfrac{CN_0}{N}\sum_{\substack{j\in \mathcal{N}\setminus i,\\ c(\sigma_j)=c(\sigma_i)}}(\dfrac{1}{\tilde{P_i}}+\dfrac{1}{\tilde{P_j}})
 = \dfrac{\sum\limits_{\substack{j\in \mathcal{N}\setminus i,\\ c(\sigma_j)=c(\sigma_i)}}f_{ji}}{ \tilde{P_i}} + \sum_{\substack{j\in \mathcal{N}\setminus i,\\ c(\sigma_j)=c(\sigma_i)}}\dfrac{f_{ij}}{\tilde{P_j}} + \dfrac{2CN_0}{N}\sum_{\substack{i\in\mathcal{S}\subset\mathcal{N},\\\mathcal{S}:\forall i\in \mathcal{S}\\ c(\sigma_i)=c}}\dfrac{1}{\tilde{P_i}}\\
\end{split}
\end{equation}
\end{strip}

where $\mathcal{S}$ denotes the set of WBSs whose working channel is the same with WBS $i$.

Now we are going to have a look at the \textit{potential} of the network.
According to the expression of Rosenthal's potential in Formula~\ref{2:Rosenthal_potential_newdelay}, the potential is accumulated by adding the players' cost sequentially, in particular, the value which is added is the cost that player experiences when it starts to use the relevant resource, and the value is not changed when other players come to use that resource.
Back to our problem, for two WBSs $i,j\in \mathcal{S}$, we assume WBS $i$'s index is smaller than $j$'s index, then the potential increased by $i$ using the resource $\{i,j,c\}$ is 0 according to Formula~\ref{6}, and the increase brought in by $j$ using the resource $\{i,j,c\}$ is $g_{(i,j,c)}(2)+g_{(j,i,c)}(2)$. 
In other words, for each interfering pair of WBSs, only the WBS with bigger index contributes to the potential. 
Note that the summation of one WBS's congestion is related to its index. 
Then the total potential is, 
\begin{equation}
\label{allPotential}
\begin{split}	
&G(\sigma) 
 =\sum\limits^{}_{r\in \mathcal{R}} \sum\limits^{n_r(\sigma)}_{i=1} g_r(i)  =\sum\limits_{i\in \mathcal{N}} \sum\limits^{}_{r\in \sigma_i} g_r(n_r^i(\sigma))\\
%& = \sum\limits_{i\in \mathcal{N}}\dfrac{\sum\limits_{\tiny\substack{j\in \mathcal{N}, j\ne i,\\ c(\sigma_j)=c(\sigma_i)}}\tilde{f_{ji}}}{\tilde{P_i}} + \sum\limits_{\tiny\substack{\mathcal{S}\backepsilon i, \mathcal{S}\subset \mathcal{N},\\ \forall j\in \mathcal{S}, j\neq i,\\ c(\sigma_j)=c(\sigma_i)}}  (\dfrac{\mid\mathcal{S}\setminus 1 \mid}{C} \frac{N_0}{\tilde{P_i}})
& = \sum\limits_{i\in \mathcal{N}}\dfrac{\sum\limits_{\tiny\substack{j\in \mathcal{N}, j\ne i,\\ c(\sigma_j)=c(\sigma_i)}}f_{ji}}{\tilde{P_i}} + \dfrac{CN_0}{N}\sum\limits_{\tiny\substack{\mathcal{S}\subset \mathcal{N},\\ \forall i\in \mathcal{S}, c(\sigma_i)=c}} \mid \mathcal{S}\mid   \sum\limits_{\tiny\substack{i\in \mathcal{S}}}\frac{1}{\tilde{P_i}}
\end{split}
\end{equation}


When players minimize their utilities (cost or potential)~\ref{cost1player}, the total potential~\ref{allPotential} in the secondary network sdecreases monotonically before reaching a Nash equilibrium. 
Players' greedy update in the game to minimize its cost Function~\ref{cost1player}, which ceases finally in pure Nash Equilibrium. The strategy and cost function of players in the game is transplanted as Algorithm \ref{whitecatalgo} and utility Function \ref{utility} respectively.


\subsubsection{Potential in the Congestion Game and the sum of Utilities}
It is interesting to know, whether the sum of the final utilities of all WBSs is exactly the same with the potential~\ref{allPotential} during the convergence process.
The answer is, they are identical when $N_0$ is zero, and there will be a minor difference when $N_0$ is not zero.
Recall the target objective we want to minimize is,
\begin{equation}
\label{compare}
\begin{split}	
\sum_{i\in \mathcal{N}}\dfrac{f_i}{\tilde{P_i}}
& = \sum\limits_{i\in \mathcal{N}}\dfrac{\sum_{\tiny\substack{j\in \mathcal{N}, j\ne i,\\ c(\sigma_j)=c(\sigma_i)}}f_{ji}+N_0}{\tilde{P_i}}\\
& = \sum\limits_{i\in \mathcal{N}}\dfrac{\sum_{\tiny\substack{j\in \mathcal{N}, j\ne i,\\ c(\sigma_j)=c(\sigma_i)}}f_{ji}}{\tilde{P_i}} + \sum\limits_{i\in \mathcal{N}}  (\dfrac{N_0}{\tilde{P_i}})\\
\end{split}
\end{equation}
We notice that only the last items of the objective~\ref{compare} and the potential of the congestion game~\ref{allPotential} are different.
When $N_0=0$, the potential is exactly the same with the object we want to minimize.
When $N_0\neq 0$, if channels are evenly distributed and there is $C/N*\mid \mathcal{S}\mid = 1$, then Formula~\ref{compare} and \ref{allPotential} are also the same.
In both cases, the sum of utilities \ref{compare} decreases monotonically with every update of WBSs before the system reaches Nash Equilibrium.
%
When $N_0\neq 0$ and Formula~\ref{compare} and \ref{allPotential} are thus different, the monotonicity on the decrease of sum of utilities \ref{compare} is not perceived, whereas the system will still cease to NE.

Based on above analysis, we can see the assumption that each WBS only occupies one channel can be easily removed.
By assuming multiple WBSs are allocated at one WBS's location while each WBS works on one distinct channel, then the proof on convergence of whiteCat can be applied directly to this case.
Note that the convergence of the game is independent on the the concrete form of the cost function. 
We adopt the function \ref{cost1player} to let the potential of the game be the same with the total utility of all WBSs, so that by executing Algorithm~\ref{whitecatalgo}, the system objective experiences a monotonic decreasing process before the system reaching NE.
The algorithm has potential to solve many other problems, where one user's decision affects others.
In this case, the utility of one user can be formulated to incorporate the information of its own utility and others', then the congestion game theory can be used to analogize.
%Hence, WhiteCat scheme provides a prototype for the problems where the interaction among users are asymmetric.



\subsubsection{Communication Overhead of WhiteCat}

The problem of channel allocation with different and fixed transmission power is NP hard.
WhiteCat is a distributed scheme but certain information of the other WBSs is needed.
The centralized base station is piggybacked to provided the needed information.
As to one WBS, the number of such inquiries is the number of steps before convergence.

In our formulated congestion game, a player $i$ is allowed to access up to $(N-1)$ resources in the same time, \ie $\{i, j_1, c(i)\}, \{i, j_2, c(i)\} \cdots \{i, j_{N-1}, c(i)\}$, thus the upper bound of converge steps can not be obtained from the conclusion~\ref{2:Rosenthal_potential_newdelay} for singleton congestion game.
But our problem is special because for each resource, the possible number of players allowed to use each resource is either 2 or 0.
Thus we can refer the method used in Section~\ref{game} to analyse the times of updates for our problem.
Firstly, we sort the cost values in increasing order.
Although a WBS 
\begin{equation}
\label{2:Rosenthal_potential_newdelay}
\begin{split}
\phi(\sigma) 
& =\sum\limits^{}_{r\in \mathcal{R}} \sum\limits^{n_r(\sigma)}_{i=1} g_r(i) \leq \sum\limits^{}_{r\in \mathcal{R}} \sum\limits^{n_r(\sigma)}_{i=1} n \leq n^2m
\end{split}
\end{equation}

The upper bound of total update steps is $2n^2$, thus averagely, the upper bound of update steps for each WBS is $2n$.


\section{Performance Evaluation}
\label{simulation}
In this section, we will investigate the centralized optimization and distributed schemes with different amount of channels.
As comparison, we implement the centralized scheme proposed in~\cite{ReAlloTVWS14DySPAN}, which is designed complying with FCC regulations.
Some adaptions are made to comply with the ECC rules, 1) When a channel is chosen, the WBS transmits with the maximal permitted transmission power permitted on that channel instead of an identical power for all the WBSs; 2) Auxiliary circle is introduced into the scheme, and the SINR is not the quotient of the transmission power and the interference on the WBSs' locations, but the power of signal and interference on the auxiliary circles.

The evaluation setting is as follows.
A square area which is 60km x 60km is divided evenly into 16 square blocks.
There is one WBS sitting in the middle of each block,  where its end terminals are distributed within the same block.
There is a 20km wide rim area around the square area, where the critical points for the DTV receivers are randomly located.
The locations of WBSs and TV contours are illustrated in Fig.~\ref{sim:layout}.
WBSs' locations are fixed, but the end terminals, and the sequence for WBS to update are randomly decided in each run.
In each run the critical points for the digital TV service are located at different places, so that the power-channel map for every WBS is different in different runs.
Simulation results are averaged over 50 runs.

\begin{figure}[h!]
  \centering
  \includegraphics[width=0.5\linewidth]{layout.pdf}
  \caption{Layout of WBSs and TV contours}
  \label{sim:layout}
\end{figure}

Some parameters are listed in Table~\ref{6}.

\begin{table}[!h]
\centering
\begin{tabular}{|l|r|}
  \hline
  Number of channels 						& 4 \\
  Number of WBSs							& 9, 16\\
  Noise 									& $10^{-13}$ Watt \\ % -90dbm
  Side length of area for locate WBSs		& 60km\\
  Auxiliary circle radius 	& 1km \\
  Inf. threshold on critical point 		& $10^{-8}$, 5x$10^{-8}$Watt \\ % -67dbm
  Path loss factor 							& 2 \\
  Standard deviation in flat shadowing		& 8\\
  Number of end terminals per cell 		& 10 \\
  Min. WBS Tx power 			& 1 Watt \\
  Max. WBS Tx power			& 4 Watt \\
  Number of simulation runs & 50 \\

  \hline
\end{tabular}
\caption{Simulation parameters}
\label{simulationparameter}
\end{table}
%\footnotetext{minimal and maximal power here denote the power level restricted by the specification of hardware.}



The first group of simulation is conducted with 9 WBSs which locate as a 3 X 3 array, there are 4 TVWS channels.
Figure~\ref{transPower} and~\ref{ECC_multiChannel_Capacity} depict the average transmission power of all the WBSs and average capacity over all the end user when working with different amount of channels.
Figure~\ref{transPower_each_cell} and~\ref{ECC_multiChannel_Capacity_each_cell} illustrate the average transmission power and capacity for each WBS over all simulation runs. 
The two proposed schemes have similar performances which increase linearly with the number of TVWS channels in use.
The greedy scheme consumes as less power as our proposed schemes when single channel is used, because many WBSs are in idle state by adopting the greedy scheme.
The average Shannon capacity is also comparable to our proposed schemes.
When more channels are allowed, our proposed schemes clearly outperform the greedy scheme in terms of achieved Shannon capacity, with the cost of high transmission power.

 \begin{figure}[h!]
    \centering
      \includegraphics[width=0.9\linewidth]{ECC_multiChannel_withComparison_Tx_1000_9WBS-crop.pdf}
    \caption{Average transmission power of all WBSs, 9 WBSs, 4 channels.}
\label{transPower}    
  \end{figure}
  
     \begin{figure}[h!]
       \centering
       \includegraphics[width=0.9\linewidth]{ECC_multiChannel_withComparison_Capacity_1000_9WBS-crop.pdf}
       \caption{Average capacity over all end terminals,  9 WBSs, 4 channels.}
	\label{ECC_multiChannel_Capacity}
     \end{figure}
%------------------------------------
 \begin{figure}[h!]
    \centering
      \includegraphics[width=0.9\linewidth]{ECC_multiChannel_withComparison_Tx_All_9_WBS-crop.pdf}
    \caption{Average transmission power of each WBS, 9 WBSs, 4 available channels. Increasing number of channels (1 to 4 channels) are used from top to bottom.}
\label{transPower_each_cell}    
  \end{figure}
  
     \begin{figure}[h!]
       \centering
       \includegraphics[width=0.9\linewidth]{ECC_multiChannel_withComparison_Capacity_All_9_WBS-crop.pdf}
       \caption{Average capacity of end terminals in each WBS's cell,  9 WBSs, 4 available channels. Increasing number of channels (1 to 4 channels) are used from top to bottom.}
	\label{ECC_multiChannel_Capacity_each_cell}
     \end{figure}


The second group of simulation is done with 16 WBSs, which is a denser scenario than group 1. 
The average transmission power of WBSs and average capacity on end users, as shown in~\ref{transPower2} and~\ref{ECC_multiChannel_Capacity2}, are similar when implying the proposed centralized and distributed schemes.
Figure~\ref{transPower2_each_cell} and Figure~\ref{ECC_multiChannel_Capacity2_each_cell} depict the average transmission power and Shannon capacity in each WBS cell.
Comparing with group 1, less transmission power is consumed by all the three investigated schemes, and meanwhile the achieved capacity is less.
This is because the network is denser in group 2 and the co-channel interference has bigger impact on the WBSs.
The greedy scheme is liable to generate more idle WBSs, as a result, both of the transmission power and Shannon capacity are less than our proposed schemes.


 \begin{figure}[h!]
    \centering
      \includegraphics[width=0.9\linewidth]{ECC_multiChannel_withComparison_Tx_1000_16WBS-crop.pdf}
    \caption{Average transmission power of WBSs,  16 WBSs, 4 channels.}
\label{transPower2}    
  \end{figure}
  
       \begin{figure}[h!]
       \centering
       \includegraphics[width=0.9\linewidth]{ECC_multiChannel_withComparison_Capacity_1000_16WBS-crop.pdf}
       \caption{Average capacity on end terminals,  16 WBSs, 4 channels.}
	\label{ECC_multiChannel_Capacity2}
     \end{figure}
%-----
 \begin{figure}[h!]
    \centering
      \includegraphics[width=1\linewidth]{ECC_multiChannel_withComparison_Tx_All_16_WBS-crop.pdf}
    \caption{Average transmission power of WBSs,  16 WBSs, 4 available channels. Increasing number of channels (1 to 4 channels) are used from top to bottom.}
\label{transPower2_each_cell}    
  \end{figure}
  
       \begin{figure}[h!]
       \centering
       \includegraphics[width=1\linewidth]{ECC_multiChannel_withComparison_Capacity_All_16_WBS-crop.pdf}
       \caption{Average capacity on end terminals,  16 WBSs, 4 available channels. Increasing number of channels (1 to 4 channels) are used from top to bottom.}
	\label{ECC_multiChannel_Capacity2_each_cell}
     \end{figure}


\section{Conclusions}
In this paper, we look into the channel allocation problem in TV white space with respect to ECC and FCC regulations respectively.
Both centralized and distributed solutions are proposed. 
In particular, we improve the SINR on the end terminals in the cells.
Under the ECC rules, the proposed centralized scheme obtains the global optimal.
In comparison, the congestion game based distributed scheme achieves comparable performance in terms of end user SINR and power consumption, but it outperforms other distributed schemes in terms of end user SINR and algorithm execution speed.
With respect to the TV spectrum usage under FCC rules, we for the first time investigate the problem of channel allocation in order to improve the SINR on end users, where the TV systems could be interfered by the aggregated interference.
Proposed centralized scheme achieves the global optimal after linear relaxizaiton, and the proposed distributed scheme is proved to converge by congestion game theory, and the simulation shows the distributed scheme outperforms the centralized scheme in terms of power consumption and SINR on end users.




\bibliography{../backmatter/myrefs}
\bibliographystyle{IEEEtran}
\end{document}